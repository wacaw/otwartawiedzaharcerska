\chapter{System zastępowy}

Kiedy  spojrzy się na  drużynę, od  razu rzuca  się w  oczy,  że nie jest ona jednolitą całością. W każdej grupie liczącej kilkanaście - kilkadziesiąt osób tworzą się  mniejsze  grupki (tak jak np. w  klasie). Stworzenie systemu zastępowego, przez Rolanda E. Philipsa było wyjściem na przeciw tworzeniu  się w drużynach właśnie takich małych grupek harcerzy, którzy wybierali sobie jednego spośród nich na przywódcę - zastępowego. Chłopcy zawierają  przyjaźnie  ze względu na: wiek, zainteresowania,  miejsce  zamieszkania, chodzenie do jednej klasy. Spełnienie trzech z tych kryteriów może być dobrym kluczem  do stworzenia  solidnego zastępu. 
	
Przy tworzeniu systemu zastępowego w drużynie nie należy nic narzucać, zastępy wytworzą się same, chłopacy sami dobiorą się w grupy, które potem staną  się zastępami. Sztucznie stworzony zastęp nie przejdzie próby czasu (długo się nie utrzyma, a ponadto, ma niewielkie szanse by skutecznie działać).

Gdy w danej grupie (przyszłym  zastępie)  znajduje  się  taki chłopak, który  zawsze ma najwięcej do powiedzenia  i  w  dodatku  pozostali  go  słuchają, wówczas z wyborem zastępowego nie ma problemu. Zastępowy powinien dbać o swoich harcerzy z zastępu,  a o rozwój zastępowych dbać powinien drużynowy poprzez prowadzenie zastępu zastępowych (w którym zastępowym jest drużynowy).

Praca zastępu zastępowych  nie powinna znacząco różnić się od pracy poszczególnych zastępów  w  drużynie. Dla wielu zastępowych zbiórki zastępu zastępowych, pod wodzą  drużynowego, mogą być swoistym poligonem doświadczalnym przed zbiórkami  poszczególnych zastępów. Zbiórka zastępu zastępowych na  oczątku tygodnia da możliwość zastępowym powtórzenia jej w  pozostałych dniach już ze  swoimi  zastępami.

System zastępowy nie oznacza tylko, że w drużynie istnieją zastępy, ale że drużynowy prowadzi drużynę przez zastęp zastępowych.

\begin{aquote}{Roland E. Philips}
  System  zastępowy  nie  jest  jedną  z  wielu  metod organizowania  pracy  skautowej, lecz jest  on  jedyną  metodą.
 \end{aquote}
