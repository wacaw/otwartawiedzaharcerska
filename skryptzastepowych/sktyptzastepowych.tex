\documentclass[10pt,polish,a5paper,final,twoside,fleqn,onecolumn]{skryptzast}
\usepackage[top=1.5cm, bottom=1.5cm, left=1cm, right=1cm]{geometry}
\usepackage[utf8]{inputenc}
\usepackage[OT4]{fontenc}

\usepackage{tikz}
\usepackage{tikz-qtree}
\usepackage{enumitem}

\usepackage{datetime}
\usepackage{multicol}
\usepackage[widespace]{fourier}   % Use Utopia font set
\usepackage[scaled=0.875]{helvet} % Use Helvetica for sans-serif
\renewcommand{\ttdefault}{lmtt}   % Use Lucida Mono for monotype
\usepackage{booktabs}             % Booktabs for tables
\usepackage{multirow}
\usepackage{algorithm2e}
\usepackage{placeins}
\usepackage{amsmath}
\usepackage{graphicx} 
\usepackage{floatflt}
\usepackage{wrapfig}
\usepackage{xspace} 
\usepackage[style=long,nonumberlist]{glossaries}

\usepackage{amsthm} % pushQED, popQED
\newenvironment{aquote}[1]{%
  \pushQED{#1}%
  \begin{quote}
}{%
  \par\nointerlineskip\noindent\hfill(\popQED)%
  \end{quote}%
}

\title{Skrypt kursu zastępowych}
\author{Praca grupowa}
\date{data utworzenia: \today}

\hyphenation{hu-f-ce no-we lwo-wskim spa-do-chro-niar-stwo szy-ty m-o-ż-e} 

\begin{document}
\maketitle
\begin{figure}[h]
\centering
  \includegraphics[width=6cm]{grafiki/lilijka.jpg}
\end{figure}
\bigskip 
\begin{center}
\textbf{ZHR}

III Poznański Hufiec Harcerzy ,,Horyzont''

Kurs Zastępowych \textbf{,,WSCHÓD I''}
\end{center}

\clearpage

\begin{aquote}{Robert Baden Powell}
Gdybym dziś mógł wybierać miejsce, które chciałbym zajmować w ruchu, to chciałbym być zastępowym.
\end{aquote}
 
\begin{aquote}{Roland E. Philipsl}
System zastępowy nie jest jedną z wielu metod organizowania pracy skautowej, lecz jest  on jedyną metodą.
\end{aquote}
\cleardoublepage
\pagenumbering{Roman}\pagestyle{skryptzast}%
\tableofcontents* \cleardoublepage%

\mainmatter
\chapter{Wstęp}

Czuwaj!
\begin{floatingfigure}[l]{3cm}
  \includegraphics{grafiki/intro.png}
\end{floatingfigure}
Niniejszy skrypt zawiera informacje, które z pewnością przydadzą Ci się w prowadzeniu zastępu. Można go nazwać czymś w rodzaju ściągi. Nie bój się do niego zaglądać, gdy o czymś zapomnisz. Śmiało notuj i podkreślaj te elementy, które są dla Ciebie ważne. Wiedz, że Kurs Zastępowych ''Wschód'' oraz ten notatnik to tylko wsparcie. Najważniejszy jesteś Ty, Twój zapał i chęć działania - to one sprawią, że Twoje zbiórki będą wyjątkowe a atmosfera w zastępie niepowtarzalna. 

Żeby być dobrym zastępowym, musisz być dobrym harcerzem. Pewnie zadałeś sobie pytanie, po co to całe harcerstwo? Trzeba wiedzieć po co coś robię i dlaczego. To są podstawy. Pamiętaj zatem: celem naszego pobytu w ZHRze jest wychowywanie metodą harcerską --- w myśl przyrzeczenia i prawa harcerskiego dobrych ludzi. Wszystko inne jest  środkiem do osiągnięcia tego celu. Przypomnijmy sobie fundamenty i podstawy: prawo i przyrzeczenie harcerskie.



\textbf{Przyrzeczenie harcerskie}%wersja ZHR

Mam szczerą wolę całym życiem pełnić służbę Bogu i Polsce, nieść chętną pomoc bliźnim i być posłusznym Prawu Harcerskiemu.


\textbf{Prawo harcerskie}%wersja ZHR
\begin{enumerate}[noitemsep,nolistsep] 
\item Harcerz służy Bogu i Polsce i sumiennie spełnia swoje obowiązki.
\item Na słowie harcerza polegaj jak na Zawiszy.
\item Harcerz jest pożyteczny i niesie pomoc bliźnim.
\item Harcerz w każdym widzi bliźniego, a za brata uważa każdego innego harcerza.
\item Harcerz postępuje po rycersku.
\item Harcerz miłuje przyrodę i stara się ją poznać.
\item Harcerz jest karny i posłuszny rodzicom i wszystkim swoim przełożonym.
\item Harcerz jest zawsze pogodny.
\item Harcerz jest oszczędny i ofiarny.
\item Harcerz jest czysty w myśli, mowie i uczynkach, nie pali tytoniu i nie pije napojów alkoholowych.
\end{enumerate}


\input{02-historia.tex}
\chapter{System zastępowy}
\section{Wstęp}
\begin{wrapfigure}{l}{4cm}
  \begin{center}
    \includegraphics[width=4cm]{grafiki/szpaler.png}
  \end{center}
\end{wrapfigure}Kiedy  spojrzy się na  drużynę, od  razu rzuca  się w  oczy,  że nie jest ona jednolitą całością. 
W każdej grupie liczącej kilkanaście - kilkadziesiąt osób tworzą się  mniejsze  grupki (tak jak np. w  klasie). 
Stworzenie systemu zastępowego, przez Rolanda E. Philipsa było wyjściem na przeciw tworzeniu  się w drużynach właśnie takich małych grupek harcerzy, którzy wybierali sobie jednego spośród nich na przywódcę - zastępowego. 
Chłopcy zawierają  przyjaźnie  ze względu na: wiek, zainteresowania,  miejsce  zamieszkania, chodzenie do jednej klasy. Spełnienie trzech z tych kryteriów może być dobrym kluczem  do stworzenia  solidnego zastępu. 
	
Przy tworzeniu systemu zastępowego w drużynie nie należy nic narzucać, zastępy wytworzą się same, chłopacy sami dobiorą się w grupy, które potem staną  się zastępami. Sztucznie stworzony zastęp nie przejdzie próby czasu (długo się nie utrzyma, a ponadto, ma niewielkie szanse by skutecznie działać).

Gdy w danej grupie (przyszłym  zastępie)  znajduje  się  taki chłopak, który  zawsze ma najwięcej do powiedzenia  i  w  dodatku  pozostali  go  słuchają, wówczas z wyborem zastępowego nie ma problemu. 
Zastępowy powinien dbać o swoich harcerzy z zastępu,  a o rozwój zastępowych dbać powinien drużynowy poprzez prowadzenie zastępu zastępowych (w którym zastępowym jest drużynowy).

Praca zastępu zastępowych  nie powinna znacząco różnić się od pracy poszczególnych zastępów  w  drużynie. Dla wielu zastępowych zbiórki zastępu zastępowych, pod wodzą  drużynowego, mogą być swoistym poligonem doświadczalnym przed zbiórkami  poszczególnych zastępów. Zbiórka zastępu zastępowych na  oczątku tygodnia da możliwość zastępowym powtórzenia jej w  pozostałych dniach już ze  swoimi  zastępami.

System zastępowy nie oznacza tylko, że w drużynie istnieją zastępy, ale że drużynowy prowadzi drużynę przez zastęp zastępowych.

\begin{aquote}{Roland E. Philips}
  System  zastępowy  nie  jest  jedną  z  wielu  metod organizowania  pracy  skautowej, lecz jest  on  jedyną  metodą.
 \end{aquote}
 
 
\section{Zastęp}
\begin{wrapfigure}{l}{3cm}
  \begin{center}
    \includegraphics[width=3cm]{grafiki/zastep.png}
  \end{center}
\end{wrapfigure}No, a co to jest ten zastęp? - zastęp jest to grupa chłopców w mniej więcej jednakowym wieku, o tych samych zainteresowaniach i pochodzących z  tego samego środowiska (szkoła - klasa, osiedle - podwórko). 
Liczba  chłopców  w  zastępie  powinna  być około 6 - 8. Pozwala to na dobrą pracę zastępu i oddziaływanie zastępowego  na swoich harcerzy (stopnie, sprawności, zabawa, wykonywanie różnych zadań, wspólne życie obozowe). 
Na czele zastępu stoi zastępowy. 
Jest to chłopiec  o najmniej w tym samym  wieku co członkowie zastępu. Wybija się on spośród nich dzięki swoim zdolnościom przewodzenia w grupie. 
Powinien być wybierany przez zastęp. 
Zastępy mogą być jednopoziomowe, gdy są nim chłopcy w jednym wieku, z jednej akcji naborowej lub wielopoziomowe, gdy co roku, ktoś z zastępu przechodzi do drużyny wędrowniczej, a na jego miejsce dochodzi ktoś nowy.

Pamiętajmy,  że zastęp jest najistotniejszą jednostką naszej organizacji. 
To od zastępu zależy poziom drużyny. 
Każdy zastępowy powinien wiedzieć, że w zastępie wyłaniają się nowe jednostki mogące dalej pozytywnie wpływać  na  rozwój  drużyny.

Silne zastępy to silna organizacja, a silna organizacja to miejsce kształtowania obywateli do  służby dla naszej Ojczyzny.
	
\section{Zastępowy}
Czas teraz zastanowić się, kto to właściwie jest zastępowy. Kim Ty masz być? Zastępowy to ktoś szczególny. To harcerz, który obok drużynowego ma do zrobienia najwięcej w drużynie. Jest to służba, trudna służba, która jednak dobrze wykonywana jest bardzo owocna. \begin{wrapfigure}{r}{4cm}
  \begin{center}
    \includegraphics[width=4cm]{grafiki/kop.png}
  \end{center}
\end{wrapfigure} 
	Zastępowy jest jeden a chłopaków wielu. To nieprawda. Zastępowych jest trzech, a wszystko to w jednej osobie.
\begin{itemize}
\item \textbf{Pierwszy zastępowy  to  wzór.}
\item \textbf{Drugi zastępowy to wódz.}
\item \textbf{Trzeci jest  starszym bratem.}
\end{itemize}


Trzy twarze jednego, bardzo ważnego człowieka. Jest wzorem, bo taki, jaki będzie on, tacy będą jego harcerze z zastępu. Wódz - znaczy przewodnik, facet, który ma wśród chłopaków siłę przebicia, słowem odpowiedzialny młody człowiek. Jeżeli zastępowy będzie kumplem, pomocnikiem, opiekunem to będzie starszym bratem. Dzięki tym trzem twarzom zastępowy może samodzielnie prowadzić zastęp i mieć tak wielki wpływ na chłopaków, że może zrobić z nich morowych chłopaków.
	Dobry  wódz musi pamiętać o tym, że niezbędna jest praca ze wszystkimi  harcerzami  ze  swojego  zastępu. Tylko dzięki takiej pracy będą najlepsi.  Tylko wówczas, gdy będą razem zdobędą szczyty.
	Zastępowy koniecznie chce być w swojej pracy najlepszy. Tu objawia się zastępowy - wzór. Musi pokazać swojemu zastępowi, że potrafi  słuchać i wykonywać rozkazy bez względu na to, czy jest obecny  drużynowy czy nie. Nieodzowne jest przodowanie zastępowego w zdobywaniu stopni i sprawności. Zastępowy - wzór sprawdza się w każdych warunkach i dlatego chłopcy pójdą za nim bez zbędnych tłumaczeń.

Nie można jednak zapomnieć, że (jak pisał Robert Baden - Powell): musicie nimi kierować a nie popychać ich. 

\noindent
Pamiętajcie także, że zastępowy:
\begin{itemize}\itemsep1pt

\item znawca technik harcerskich
\item  świeci przykładem: styl harcerski, pogoda ducha, nauka, dom, kultura, pomoc bliźnim
\item  dba o wszystko w  zastępie:  począwszy od  tego czy jego harcerze noszą pas  w szlufkach, przez tworzenie nowych gier, wyjazdy na wycieczki, czy biwaki, aż do zdobywania sprzętu na obozy.
\item  prowadzi zastęp wg planu, który wymyślił wraz zastępem - pisze więc mądry plan działania i konsekwentnie go realizuje, aż osiągnie cel. Prowadzi też książkę lub zeszyt zastępowego (nie na  pokaz  tylko dla samego  siebie), gdzie ma listę swoich harcerzy, notatki o nich, sieć alarmową, plan działania, plany zbiórek, comiesięczny rachunek sumienia (co się udało a co nie), decyzje Rady Zastępu i Rady Drużyny i wszystko co jeszcze jest mu potrzebne w pracy zastępu.
\item  zna i wciąż poznaje: problemy, zdolności, zainteresowania, niepowodzenia itp. swoich harcerzy z zastępu.
\item   widzi, że każdy chłopiec z zastępu jest inny: zastęp to nie szara masa identycznych chłopaków, jeśli to zacznie dostrzegać to będzie wiedział,  że  z każdym chłopakiem trzeba pracować osobno.
\item  uczy chłopaków odpowiedzialności za siebie i  innych.
\item  nie jest dla harcerzy kapralem i ograniczeniem, lecz pociąga ich za sobą.
\end{itemize}

I jeszcze jedno:
\begin{aquote}{Hm. P. Stawiński HR}
Zastępowy  nosi  głowę  w chmurach, ale  nogami  mocno  stąpa  po  ziemi.
 \end{aquote}



\section{Duchowość zastępowego}

Zastanawiasz się czasem, czy harcerz to ktoś wierzący w Boga? – tak, to ktoś właśnie taki. Co więcej, harcerz to wzór życia duchowego, moralności – sami wiecie jak często jest z tym u nas trudno. Tym większa rola zastępowego.

W wielu domach rodzice nie przekazują już wiary, nie jest ona silna. Ty masz za zadanie być wówczas kimś, kto kieruje także duchowości swojego zastępu. Twój zastęp, to taki mały Kościół, który został przez Boga powierzony właśnie Tobie. Powinieneś dbać o to, czy chłopcy chodzą do Kościoła, do spowiedzi, czy się modlą, robią rachunek sumienia, czy starają się być lepsi każdego dnia. Wiadomo, nie można ich do tego zmuszać, ale zachęcać. \begin{wrapfigure}{r}{3cm}
  \begin{center}
    \includegraphics[width=3cm]{grafiki/duchowosc.png}
  \end{center}
\end{wrapfigure} W jaki sposób? – proste, przykładem własnym. Kiedy Ty będziesz się modlił, uczestniczył we Mszy św., kiedy będziesz lektorem i będziesz dobrym człowiekiem, oni bardzo szybko zapragną być tacy jak Ty. Wówczas szybko staną obok Ciebie przy ołtarzu, czy w kolejce do konfesjonału. Widzisz jakie to proste. Proszę też, abyś czasem modlił się za swoich chłopców w zastępie, drużynowego, to także wielka dla nich pomoc. Nie zapomnij też, ze masz obok kapelana, który może Ci zawsze pomóc.

Na obozie zaś nie zapominaj o modlitwie porannej, wieczornej, przed posiłkami, po nich, o Mszy św., o wieczornym rachunku sumienia i o tym, że Bóg jest gdzieś blisko, może nawet w drugim człowieku\ldots
\chapter{Działania zastępu}
\section{Plan pracy zastępu}
Napisanie planu pracy zastępu, to połowa sukcesu. Jeśli będziesz według niego działał, to wówczas możesz być pewny, ze żadna twoja zbiórka nie będzie nudna, a zastęp będzie ciągle się powiększał. Plan pracy składa się z:
\begin{itemize}[noitemsep] 
\item \textbf{Strony tytułowej}
\item \textbf{Ogólnej charakterystyki zastępu} --- opis sytuacji, cele długofalowe, cele na rok pracy materialne;
\item \textbf{Szczegółowej charakterystyki zastępu} --- tu zamieszczamy opis każdego przedstawiciela zastępu: wiek, szkołę, ilość lat w harcerstwie, przebieg jego służby, pasje i to, nad czym powinien popracować; charakterystyka powinna być dość szczegółowa, gdyż przecież znasz swój cały zastęp;
\item \textbf{Cele szczegółowe na rok pracy}
\item \textbf{Szczegółowy plan pracy} --- nie zapomnij, że zbiórki zastępu masz co tydzień, a raz w miesiącu zbiórkę drużyny; tu opisujesz planowany temat zbiórki i ogólnie zajęcia, jakie przeprowadzisz na niej.
\end{itemize}

\noindent
Co najważniejsze, plan powinieneś oddać do 03 IX każdego roku, aby drużynowy na podstawie planów pracy wszystkich zastępów, stworzył plan pracy drużyny. 

\section{Obrzędowść zastępu}
Każdy naród posiada swój język i własną kulturę. Właśnie to wyróżnia go spośród innych. Podobnie jest i w  harcerstwie. Mimo, że wszystkim nam  chodzi o to samo, coś sprawia że harcerze z 15 PDH, różnią się od tych z 1 ŁDH. To  coś  to  nie  tylko  barwy i numer, ale wszystko to, co jest językiem i kulturą drużyny - obrzędy.
	Po co komu obrzędy? Jest  kilka  odpowiedzi:
\begin{itemize}[noitemsep,nolistsep] 
\item  tradycja --- ludzie się zmieniają,  świat się zmieni,  ale obyczaje zostaną;
\item  język --- którym mówi się chłopakom o co tu  właściwie chodzi;
\item  dyscyplina --- ułatwia techniczne prowadzenie np. zbiórki czy ogniska;
\item  jedność --- my mamy swoje obrzędy;
\item  przyciąganie ---  tajemnica intryguje;
\item  i wiele innych rzeczy.
\end{itemize}
	Czy tylko drużyna może posiadać obrzędowość? Nie, oczywiście także zastępy muszą mieć swoje obyczaje. Ale:
-  muszą one współgrać z obyczajami drużyny
- nie może być  ich zbyt  dużo  aby  nie  przytłaczały
- muszą być nijako przy  okazji - chłopacy nie mogą całą zbiórkę wkuwać obrzędów
- nie wolno ich zmieniać bez powodu.

	Obrzędy i obyczaje  mogą  być  bardzo  różne:  od  sposobu  noszenia  proporca, przez  jakiś  znak  zastępu,  do  koloru  sznurowadeł  funkcyjnych zastępu  włącznie.  Ale uwaga! Obrzędowość nie  może być zbędnym balastem - musi pomagać, a nie  przeszkadzać.
	Kiedy chcesz stworzyć nowy obrzęd,  zastanów się po co chcesz to  zrobić. 
	
	
\section{Zbiórka zastępu}

Pytasz sam siebie: po co nam właściwie te zbiórki? 
Niezbyt kulturalnie można by odpowiedzieć pytaniem: a po co Wam właściwie harcerstwo?

Po to, żeby co dzień być trochę lepszym, trochę więcej umieć, trochę więcej wiedzieć, żeby  w końcu spędzić  sensownie czas, a to wszystko w kontekście celu wychowawczego ZHRu, o którym mówiliśmy już wcześniej.
	
No więc zostawił Cię Twój  okrutny drużynowy z zadaniem  przygotowania  zbiórki. 
Im bliżej terminu tym bardziej się denerwujesz. 
A może by tak nie pójść? 
Udać chorego? 
Eee\ldots
	
Siadasz w końcu nad kartką i rozmyślasz co by tu zrobić. Coś tam piszesz i  idziesz na zbiórkę. 
	
A może użyć  na  początek  takiego schematu:

Po co, czyli jaki \textbf{SENS} ma ta zbiórka? 
(kiedy chłopacy wrócą ze zbiórki muszą jednym zdaniem powiedzieć co robili: pełnili służbę w domu dziecka, grali w piłkę, zdobywali sprawności, odwiedzali super człowieka, harcowali po lesie) 
Pamiętaj, że oni zaraz wyczują, że zbiórka jest  tylko posklejana bez ładu z różnych gier.

Musi być \textbf{POMYSŁ}. czasem szalony, nierealny, ale porywający. Zrobić grę w niedostępnej części muzeum, odwiedzić drużynowych Twojej drużyny z ostatnich dziesięciu lat, zaprosić komandosa, urządzić bieg na podstawie aktualnego filmu z Jamesem Bondem, wleźć na czubek ratusza\ldots

Żeby to przeprowadzić musi być \textbf{PLAN}, czyli jak wprowadzić w życie mój (nasz) pomysł.
Gra  wg filmu Na  krawędzi z Sylwestrem Stallone:
\begin{enumerate}[noitemsep,nolistsep] 
\item spotkanie na Cytadeli;
\item szukanie walizki z milionem dolarów wg planu rozdanego wcześniej;
\item pokonanie przeciwników podczas podejścia pod najwyższą możliwą górę (a jak  się  ich  pokonuje?);
\item odnalezienie walizki z \ldots czekoladą w środku.
\end{enumerate}
A jaki był  sens tej zbiórki? 
O choćby ćwiczenia fizyczne podczas gonitwy, nauka czytania planu, podchodzenia, uczciwa walka itd.
	
Dużo czasu  zajmuje \textbf{PRZYGOTOWANIE} , ale nie zawsze sam musisz  rysować plan, przynosić walizkę i czekoladę, tasiemki do zrywania w pojedynkach. 
Rozdziel zadania.

Wróciłeś ze zbiórki. Podobało  się chłopakom?
Zapytaj siebie, czy gdybyś jej nie  przygotował, ale uczestniczył w takiej zbiórce przyszedłbyś na następną?
	
Może zapomniałeś o \textbf{ATMOSFERZE}? 
Ona nie zależy od terenu, miejsca. 
Ona zależy  od Ciebie! 
Traktowałeś wszystkich równo? 
Nie darłeś się na harcerzy? 
Opowiedziałeś kawał? 
To właśnie powoduje, że zbiórka nie jest lekcją!!!

\section{Zasady dobrej (Twojej) zbiórki}
Kiedy już opracujesz plan swojej zbiórki możesz sprawdzić czy zachowane są poniższe zasady. Nie przygotowuj zbiórki patrząc na tę ściągę. Czasem trzeba z któreś zasady zrezygnować na rzecz innej. Oceń swój plan i już  przeprowadzoną zbiórkę z tą listą. Wkrótce Twoje plany same się z nią zgodzą - nie będziesz już potrzebował ściąg.
\begin{description}


\item
[ZASADA 1 ZBIÓRKA MA LOGICZNY  CIĄG]
Jeden element zbiórki wypływa z drugiego. 
Z gawędy gra. 
Z gry - nauka  technik. 
Z technik - znów gra. 
Z gry - piosenka i tak dalej. 
Jeśli o tym zapomnisz, to ciągle będziesz zerkał do kartki z przygotowaną zbiórką (lekcje?),  a chłopcy zawsze to wyczują. 
Oni nie lubią drętwych zbiórek.
\item 
[ZASADA 2 CHARAKTER ZAJĘĆ MUSI  SIĘ ZMIENIAĆ]
Kiedy planujesz zbiórkę pamiętaj, że zbyt jednolite zajęcia nużą. 
Szybciej  spokojne, ale ruchowe także. 
Więc tak pozmieniaj kolejność zajęć  żeby mieszały się powaga (gawęda) z wesołością (piosenka), cisza (kominek) z hałasem (okrzyk  zastępu) spokój (praca rąk) z ruchem (gra ruchowa, turniej) dyscyplina (musztra, apel)  z luzem (zwiad,  gra).
\item
[ZASADA 3 ZBIÓRKA MUSI MIEĆ TEMPO]
Zanim ktokolwiek zapyta: czemu ta przerwa?, Ty musisz już przedstawić następny punkt programu. W czasie prowadzenia  gry,  mówienia gawędy i ćwiczenia węzłów, patrz czy  chłopcy się nie kręcą, nie zaczynają opowiadać kawałów. 
Wtedy  przerwij nawet jeśli nie zrobiłeś wszystkiego. 
Kończ  zbiórkę zanim się znudzą, wtedy chętniej przyjdą na następną.
\item
[ZASADA  4 ZASTĘPOWY JEST Z NAMI]
Nie siedź  z boku przyglądając się harcerzom. 
Weź udział w grze, najgłośniej się wydzieraj w okrzyku zastępu. 
Nie baw się w dyrektora, bądź jednym  z nich.
\item
[ZASADA  5 ZBIÓRKA  MA  CZTERY  STAŁE  ELEMENTY]
Obrzęd powitania, gawęda - to może być  wizyta fajnej osoby, dyskusja lub ciekawie opowiedziana historia - 5 minut to dosyć, nigdy nie przekraczaj 10 minut (5 minut gawędy to 20 minut jej przygotowywania w domu!), Rada Zastępu - tu  przekażesz sprawy bieżące,  tu  rozwiążecie wspólny problem, pomyśl nad obrzędem rozpoczęcia (może np. postawicie między Wami proporzec zastępu?), niech Rada trwa nie dłużej niż 15 minut, obrzęd pożegnania -  może jakiś specjalny Wasz krąg, krótka piosenka, może coś jeszcze innego?
\item
[ZASADA  6 COŚ NOWEGO I COŚ STAREGO NA KAŻDEJ ZBIÓRCE]
To chyba jasne? 
Przez przypomnienie znanych rzeczy, łączysz zbiórkę z poprzednią, wprowadzasz klimat. 
Możesz też utrwalić, przypomnieć to, co łatwo  się  zapomina.
\item
[ZASADA 7 NIE MA ZBIÓRKI BEZ INICJATYWY CHŁOPAKÓW]
Ciesz się kiedy chłopcy mają własne pomysły, chcą nagle przypomnieć starą grę, ulubioną piosenkę, zmienić przeprowadzaną przez Ciebie zabawę. 
To znaczy, że traktują zbiórkę jak swoją, chcą żeby była jak najfajniejsza. 
Nie możesz ich wiecznie pouczać, dyrygować  nimi. 
Jeśli w czasie zbiórki zdarzy się coś niezwykłego nie trzymaj się kurczowo napisanego  planu, ale wykorzystaj okazję żeby coś nowego poznać, zobaczyć, a zwłaszcza zrobić dobry  uczynek.
\item
[ZASADA  8 ZAWSZE DZIELIMY PRACĘ]
Jeśli zawsze sam przygotowujesz zbiórkę, to w końcu wyczerpią Ci się pomysły. Poza tym nauczysz harcerzy czekania na to, co będzie zamiast współtworzenia zbiórek.

Daj im zadania: przynieść tekst piosenki, życia do gry w terenie, busolę itd. Jeśli trzeba zaprosić gościa niech załatwią to chłopacy. Tylko pamiętaj: jeśli przed zbiórką nie zadzwonisz z pytaniem o wykonanie zadań, możesz się znaleźć w sytuacji, w której zabraknie Ci głównego punktu zbiórki!
\item
[ZASADA  9 NIESPODZIANKĘ ZAWSZE SIĘ WSPOMINA]
Staraj się przynajmniej co jakiś czas zaskoczyć mile chłopaków. Nie daj  im  pomyśleć,  że  wpadasz  w  rutynę i przyzwyczajenia.
	
	Co to może być? Musisz już sam wymyślić: ciekawą osobę, film video,  zbiórkę z zastępem harcerek...
\end{description}
	Że co? Że niby dużo tej teorii? Jeśli parę razy przygotujesz zbiórkę według  tych  paru zasad, same Ci  wejdą w krew i nawet nie będziesz wiedział kiedy i jak. Ale nie zaszkodzi  o jakiś czas tu zerknąć i  sobie  przypomnieć.
\begin{itemize}

\item ZASADA 1 - LOGICZNEGO  CIĄGU
\item ZASADA 2 - PRZEMIENNOŚCI ELEMENTÓW ZBIÓRKI
\item ZASADA 3 - TEMPA
\item ZASADA 4 - ZASTĘPOWY TEŻ JEST Z NAMI
\item ZASADA 5 -  CZTERECH  STAŁYCH  ELEMENTÓW ZBIÓRKI
\item ZASADA 6 -  COŚ  STAREGO I  COŚ NOWEGO NA ZBIÓRCE
\item ZASADA 7 -  SAMODZIELNOŚCI I INICJATYWY CHŁOPCÓW
\item ZASADA 8 - PODZIAŁU PRACY
\item ZASADA 9 - POZYTYWNEGO ZASKOCZENIA I NIESPODZIANKI

\end{itemize}
Na koniec coś co może Tobie  ułatwić planowanie zbiórek. To formy pracy:

Formy podstawowe:
-  ognisko	-  gawęda	- pląs 		- ćwiczenie
- okrzyk 	-  pokaz		- kominek	- zwiad
-  śpiewy	-  zadanie	- popis		- quiz
-  szkolenie	-  bieg		- turniej		- dyskusja
- rozmowa	-  obrzędy	- festiwal	- zawody
-  apel		-  musztra	- praca rąk	- próba
- gra		- harc		- alarm		-  podchody
- konkurs	- herbatka	- służba
- zadania zespołowe		- zadania międzyzbiórkowe
- działania społeczne			

Formy złożone :
-  zbiórka	-  obóz		-  biwak		-  włóczęga
-  zlot		-  kolonia	-  wycieczka	-  zimowisko
-  rajd		-  złaz

\chapter{Stopnie i sprawności}
\section{Stopnie harcerskie}
\begin{wrapfigure}{l}{4cm}
  \begin{center}
    \includegraphics[width=4cm]{grafiki/krzyz.png}
  \end{center}
\end{wrapfigure} 
Każdy z nas zdobywa stopnie. Od chwili gdy przychodzimy na pierwszą zbiórkę, zaczynamy mozolnie wspinać się po drabinie harcerskiego życia, zdobywając szczebel po szczeblu kolejne stopnie. Pokazują one nie tylko ile drogi za nami,  ale przede wszystkim są drogowskazami w marszu do szczytu  ideałów.
	Nasz stopień staje się tak prawie ważny jak imię i nazwisko (wszak wymawiane jednym tchem), starajmy się abyśmy mogli być  z  niego dumni! Kiedy dostanę Krzyż, czy stopień? To pytanie stawiane często starszym druhom przez nowych chłopców w drużynie.  Pamiętajmy Krzyża Harcerskiego, stopnia się nie dostaje, trzeba go zdobyć i bynajmniej nie gadaniem o nim, tylko dobrze przygotowaną i zrealizowaną próbą. Zdobywanie nowego stopnia możemy podzielić na kilka etapów.  Oto one :


1.
Harcerz  przychodzi  do  drużynowego  albo Kapituły stopni (jeśli taka działa w drużynie) i zgłasza swoją  chęć  zdobywania  stopnia. Oczywiście  ktoś może mu to podpowiedzieć,  ale to musi być jego decyzja

2.
Chłopiec  przez  okres  kilku  miesięcy  pracuje  nad  sobą,  w   myśl  wskazówek drużynowego  lub  opiekuna  stopnia, stara  się  dostosować do wymagań stopnia.

3.
Drużynowy  opiekun lub Kapituła stopni  po  ocenie poprzedniego  etapu, uzgadnia  z harcerzem próbę  składającą się  z  ok. 10  zadań obejmujących  różne  dziedziny. Próba  jest  bardzo indywidualna (tak  jak  nie  ma  dwóch Jarków  Wiśniewskich, tak  nie ma dwóch takich samych prób), jak  też  bardzo  konkretna (więc nie ma  zadań  w  stylu powiadomił kogoś  w  ważnej sprawie, ale  np. powiadomiłem  druha Jacka  o  biwaku w  dniu  10 X). Wielką pomocą jest tutaj książeczka z regulaminem stopni harcerskich, która na pewno ma twój drużynowy, a którą może by było warto, abyś posiadał. Tam możesz zaznajomić się z ideą stopnia i wymaganiami na jego zdobycie.

4.
Otwarcie  próby  ogłasza  drużynowy  rozkazem (dla stopni HO - hufcowy, a dla stopnia HR - komendant  Chorągwi), a może  być  zamknięta   z  wynikiem  pozytywnym lub  negatywnym. Próba  trwa  ok. 3 - 12  miesięcy.

5.
Ostatni etap to próba końcowa trwająca 1 do 3 dni. Jest to jakiś wyczyn pokazujący, że ten  harcerz  siłą  charakteru, duchem i zaradnością  udowodnił,  że  jest  wzorowym   np. wywiadowcą.
Zdobyty  stopień   przyznaje  drużynowy, hufcowy (ćwika  i   Harcerza Orlego) lub komendant  Chorągwi (Harcerza  Rzeczypospolitej).


\section{Sprawności harcerskie}


\begin{wrapfigure}{l}{3cm}
  \begin{center}
    \includegraphics[width=3cm]{grafiki/sprawnosc.png}
  \end{center}
\end{wrapfigure} Zbliżamy się powoli do naszego szczytu Siula Grande. Zdobywanie sprawności, jest niezwykle ważnym elementem harcerskiej kariery. Oczywiście chodzi tu o karierę w  doskonaleniu  samego  siebie. Osobą, mającą nieoceniony wpływ na zdobywanie sprawności w Twojej drużynie masz Ty Druhu Zastępowy. Nie każdy harcerz, od razu wpadnie na pomysł jaką sprawność może zdobyć. Nie każdy drużynowy, znajdzie tyle czasu, aby zajmować się wszystkim w drużynie. Dlatego Roland Philips wymyślił system zastępowy. To ty swoją służbą - Druhu  Zastępowy  pomagasz  zarówno harcerzom z zastępu jak i odciążasz drużynowego.

\noindent
	Zatem w przypadku zdobywania sprawności musisz: 

1. Wiedzieć co to jest sprawność.

2. Wiedzieć jak sprawność należy zdobywać.

3. Umieć przeprowadzić próbę na sprawność.

4. Posiadać własny egzemplarz regulaminu sprawności.

5. Znać go jak najlepiej.

Ad.1
Sprawność - jest to umiejętność, którą harcerz potwierdził  wykonując konkretne dzieło, zadanie. To musi  być  coś  porządnego. Podczas zdobywania sprawności nie ma czasu na naukę. Próba  sprawności jest swojego rodzaju egzaminem.

Ad. 2
Jak  zdobyć sprawność? Najpierw zapytaj samego siebie co chcesz zrobić? Najlepiej gdy Twój pomysł pasuje do realizowanego planu pracy drużyny, obozu, czy warunków  zewnętrznych. Trudno bowiem latem zdobyć sprawność Eskimosa. Po dokonaniu wyboru musisz  przeanalizować wymagania i nie przepisywać ich, tylko ułożyć konkretne zadania. Zadania te musi jeszcze zatwierdzić drużynowy lub Kapituła. Następnie  po  zatwierdzeniu najlepiej jak potrafisz wykonujesz zadania i już zdobyłeś sprawność. Teraz tylko rozkaz drużynowego i maksymalnie 3 dni na wyszycie sprawności na rękawie munduru! To  wszystko dotyczy Ciebie. Jeżeli sprawność zdobywa harcerz z Twojego zastępu, Ty  wiedząc więcej powinieneś mu  pomóc w dojściu do otwarcia próby. Dalej jednak musi sobie radzić sam.

Ad. 3
Jako zastępowy możesz mieć powierzone  przeprowadzenie części, a nawet całej  próby. Pamiętaj, że sprawność zdobywa się konkretnym działaniem, a nie gadaniem. W takiej sytuacji trzeba być bardzo konsekwentnym i wymagającym. Taką postawę trzeba  jednak łączyć  z  życzliwością! Pamiętaj, że sprawności przestają być atrakcyjne zarówno gdy próby  są zbyt trudne  jak  i  wtedy gdy próby  są  zbyt  łatwe.

Ad. 4
Jeżeli jeszcze nie masz Regulaminu sprawności, to natychmiast zażądaj od drużynowego, by ułatwił Ci kupienie go.

Ad. 5
Zanim zaczniesz wymądrzać się na temat zdobywania sprawności, przeczytaj kilka razy regulamin (całość ze wstępem i komentarzem!).

I jeszcze jedno\ldots Pamiętaj, że jako zastępowy powinieneś mieć kilka krążków więcej na rękawie munduru, niż inni harcerze z zastępu. Pomagając innym zdobywać sprawności, nie zapominaj o swoich próbach!

Oto wymagania na dwie sprawności  jednogwiazdkowe. Spróbuj w oparciu o te wymagania ułożyć zadania do próby.
\paragraph{Łącznik:}	\begin{itemize}[noitemsep,nolistsep] 
\item  bezbłędnie  przekazał ustny meldunek lub rozkaz złożony z kilkunastu słów
 \item zaadresował  list z użyciem kodu pocztowego, wypełnił blankiet telegramu, listu  poleconego, przekazu pieniężnego
\item  zapisał wiadomość korzystając z szyfru lub alfabetu Morse’a
\item  pełnił służbę łącznika w trakcie gry terenowej, zwiadu, wycieczki, okolicznościowej uroczystości.
\end{itemize}
\paragraph{Sobieradek obozowy:}	\begin{itemize}[noitemsep,nolistsep] 
\item  zaprojektował i wykonał drobny przedmiot pionierki obozowej (drogowskaz,  ławka,  wieszak, itp.)
\item  przygotował teren pod ognisko, ułożył  je i zamaskował po nim teren
\item  w pracach pionierskich  posłużył  się: piłą, toporkiem, młotkiem, potrafi je zakonserwować po skończonej pracy
\item  wraz z zastępem wykonał wnętrze swojego namiotu
\item  w pracach pionierskich posługiwał się swoimi wymiarami (rozstaw palców długość od końca dłoni do końca łokcia itp.)
\item  rozbił samodzielnie namiot  dwuosobowy, a z zastępem dziesięcioosobowy, w razie czego okopał go.
\item  wykazał umiejętności pionierskie stosując węzły
\end{itemize}
\chapter{Techniki harcerskie}

\section{Terenoznawstwo}

Orientacja w terenie jest jedną z ważniejszych umiejętności jaką powinien posiadać każdy harcerz. Orientowanie się w terenie oprócz umiejętności zorientowania mapy ułatwia również obserwacja zjawisk przyrody. 
\paragraph{Sposoby orientacji w terenie}
\begin{description}[noitemsep,nolistsep] 

\item[Według położenia słońca] --- polega na ustawieniu się w słoneczny dzień w południe tyłem do słońca i zorientowaniu wg zasady że nasz cień pokazuje południe
\item[Według słońca i zegarka] --- jest to dość dokładna metoda ---- małą wskazówkę zegarka kierujemy ku słońcu. 
Teraz kąt między wskazówką, a cyfrą 12 dzielimy na połowę. Linia podziału wskazuje kierunek południowy (o 12:00). 
Pamiętać trzeba, że przed południem bierzemy pod uwagę kąt między małą wskazówką, a godz. 12. Po południu kąt między linią na godzinę 12, a małą wskazówkę
\item[Według gwiazdy polarnej] --- znajdujemy najpierw Wielki Wóz, a następnie przeprowadzamy przez dwie skrajne gwiazdy prostej, na której okładamy 5-krotną odległość między tymi gwiazdami. 
Na końcu tego odcinka znajduje się Gwiazda polarna wyznaczająca północ.
\item[W starych kościołach] ołtarze skierowane są w kierunku wschodnim.

\end{description}
\paragraph{Skala mapy}

Skala mapy informuje nas ile razy dana mapa została pomniejszona. Jak obliczyć jaka odległość na mapie odpowiada jakiej odległości w rzeczywistości? Jest to szalenie proste wystarczy z danej skali, np. 1:100000 skreślić 5 ostatnich cyfr. Wyjdzie nam wtedy, że 1 cm. na mapie odpowiada 1 km. w rzeczywistości. Szczegółowe mapy w Poznaniu można kupić na ul. Hawelańskiej 10 w Wojewódzkim Ośrodku Dokumentacji i Kartografii. Świetne mapy są dostępne także na stronie projektu http://geoportal.gov.pl

\paragraph{Mierzenie w terenie}

W terenie znajdują się różne rzeczy, których normalny śmiertelnik nie jest w stanie zmierzyć. No ale jako ze my jesteśmy harcerzami i dajemy sobie radę w wielu sytuacjach, to i w tym wypadku nie będzie to dla nas problem.

Pomiar szerokości rzeki - wbijamy krótką tyczkę na brzegu rzeki, i oddalamy się od niej z laską skautową na taka odległość, żeby z końca laski widzieć równocześnie koniec tyczki i krawędź drugiego brzegu.

Teraz za pomocą wzoru obliczamy szerokość rzeki:

AB= (DE x BD)/EF


2) Pomiar wysokości drzewa - sposób 1. 
Jeden z harcerzy zaznacza na lasce wyraźnie swoją wysokość od stóp do poziomu oczu, potem kładzie się na ziemi na wznak w odległości od podstawy drzewa przypuszczalnie równej wysokości drzewa. 
Drugi trzyma laskę pionowo przy jego stopach. 
Leżący przesuwa się na ziemi tak długo, aż znajdzie punkt, z którego przez miejsce zaznaczone na lasce zobaczy wierzchołek drzewa. 
Zmierzona ilość kroków od oka leżącego do podstawy drzewa na metry jest wysokością drzewa.

3) Pomiar wysokości drzewa - sposób 2. Jeden z harcerzy staje pod drzewem. 
Drugi bierze kijek i stając w pewnej odległości przy wyciągniętej ręce z kijkiem zaznacza na nim wysokość harcerza. 
Następnie przy tej samej odległości i takim samym wyciągnięciu ręki mierzy kijkiem wysokość drzewa. 
Otrzymaną wysokość drzewa dzieli przez wysokość harcerza. 
Otrzymaną liczbę trzeba już tylko pomnożyć prze rzeczywisty wzrost harcerza.

\section{Łączność}
Łączność To bardzo rozległa dziedzina. 
Nie jest to wbrew pozorom tylko alfabet Morse’a i kilka innych szyfrów. 
Łączność to komunikacja międzyludzka za pomocą różnego rodzaju urządzeń i szyfrów. 
Do łączności zaliczają się więc m.in. : wysyłanie e-maili, rozmowa telefoniczna, łączność krótkofalarska, rozmowa za pomocą alfabetu Morse’a czy przekazywanie sobie nawzajem zaszyfrowanych wiadomości. 
W poniższym rozdziale przedstawię Ci jednak zagadnienia związane z szyframi. 
	
\textbf{Alfabet Morse’a}
W 1837 roku niejaki Samuel Finley Breese Morse wynalazł telegraf elektromagnetyczny, a w 1840 stworzył dla niego specjalny alfabet telegraficzny zwany właśnie alfabetem Morse’a. 
Dzięki temu można było na duże odległości przekazywać sobie w krótkim czasie wiadomości, co było wcześniej niemożliwe. 
Obecnie alfabet Morse’a oprócz tego, że stosowany jest przez harcerzy, to wykorzystują go np. krótkofalowcy w łącznościach długodystansowych.

Alfabet ten składa się z kropek i kresek, które ułożone w odpowiedniej kolejności tworzą daną literę alfabetu. 
Żeby łatwiej było szyfrować i odczytywać wiadomości istnieje taka zasada: do każdej litery alfabetu przyporządkowywany jest odpowiedni wyraz, który dzielimy na sylaby. 
Jeśli w danej sylabie znajduje się „o” to piszemy kreskę, w przeciwnym razie kropkę.


\section{Szyfry}






\chapter{Biblioteczka zastępowego}
Spis ten nie wyczerpuje wszystkich pozycji przydatnych zastępowemu, do tej listy dołączyć należałoby wiele innych książek o tematyce historycznej, przygodowej, krajoznawczej. W biblioteczce zastępowego powinno być miejsce także dla śpiewników, map, czasopism, wydawnictw okazjonalnych itp. Biblioteka zastępowego to kopalnia pomysłów na zbiórki, wyprawy... Pamiętajcie jednak by nie zżynać dokładnie z książek, twórzcie w oparciu o to co przeczytaliście nowe, lepsze, ciekawsze formy pracy!

R. Baden-Powell - Skauting dla chłopców
R. Baden-Powell – Wskazówki dla skautmistrzów
W. Błażejewski - Z dziejów harcerstwa polskiego
A. Chmielewski - Tropy i ślady zwierząt
M. Cmielowa - Wykapka
J. Dąbrowski - Gry i zabawy w izbie harcerskiej
J. Dąbrowski - Harce zimowe
J. Dąbrowski - W świetlicy harcerskiej
J. Dąbrowski i T. Kwiatkowski - Jeden trudny rok
A. Dziewanowska, K. Rejs i Z. Zakrzewska - Gry i ćwiczenia w zastępie harcerskim
S. Gawkowski - Szkice polowe harcerza
E. Grodecka i J. Zwolakowska - Ćwiczenia i gry
A. Gromski - Harce młodzika i wywiadowcy
W. Hansen – Wilk, który nigdy nie śpi
J. Jasiński - Gry i ćwiczenia terenowe
A. Kamiński - A. Małkowski
A. Kamiński - Kamienie na szaniec
A. Kamiński - Zośka i Parasol
M. Kapiszewska - Księga Harców R.Philips - System zastępowy
A. Kazanecki - Terenoznawstwo dla harcerzy
A. Kazanecki - Z notatnikiem i busolą - zbiór gier
B. Kowalska i A. Kiewicz - Eskimos (materiały do zbiórek)
M. Kudasiewicz - Obrzędowy piec
T. Kwiatkowski - Obrzędy harcerskie
O. Małkowska - A. Małkowski
O. Nassalski - Jak pracować nad charakterem
W. Nekrasz - Pionierka harcerska
J. Parzyński - Obóz harcerski
A. Pawełek - Młoda drużyna
W. Pijanowski - Rozkosze łamania głowy
W. Pijanowski - Skarbnica gier
St. Sedlaczek - Drogowskaz Harcerza
St. Sedlaczek - Geneza skautingu i harcerstwa
St. Słysz - Gry i zabawy
St. Sosnowski i J. Stykowski - Sakwa włóczykija
W. Śliwerski - Harcerskie biegi
J. Stykowski - Na traperskiej ścieżce
J. Stykowski - Wyspa Robinsona
J. Stykowski - Zastęp zbiórka (cykl 4 częściowy: wiosna, lato, jesień, zima)
W. Szczygieł - Jak prowadzić zastęp harcerski
W. Szyrzyński - Ambulans harcerski
W. Szyrzyński - Wycieczki harcerskie
Z. Trylski - Mały podręcznik obozowania
Z. Trylski - Obozy 
L. Ungeheuer - Próby wodzów
B. Wachowicz – Wierna rzeka harcerstwa
M. Wardęcki - Harcerskie gry terenowe
A. Wasilewski - Pod totemem słońca
P. Wieczorek – Wzgórze rosiczki
R. M. Wujek - Skauting dzisiaj (gawędy o harcerstwie)
Z. Wyrobek - Vademecum skauta
Z. Wyrobek - Harcerz w polu
http://www.zhr.pl/
http://www.zastepowy.zhr.pl/
\chapter{Autorzy treści}

Marcin Gmerek ćw., 
pwd. Szymon Fiedler HO, 
pwd. Krzysztof Firlik HO, 
phm. Maciej Kaczmarek HO, 
pwd. Jarosław Kowalski HO, 
hm. Tomasz Łęcki HR, 
phm. Tomasz Nowacki HO, 
Marcin Rozmiarek ćw., 
phm. Jarosław Wagner HR, 
o. pwd. Marcin Wrzos HR 
i pwd. Wacław Łuczak HO



%\bibliographystyle{plain}
%\bibliography{bibliography}
\end{document}