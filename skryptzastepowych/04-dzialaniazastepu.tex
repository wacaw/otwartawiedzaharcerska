\chapter{Działania zastępu}
\section{Plan pracy zastępu}
Napisanie planu pracy zastępu, to połowa sukcesu. Jeśli będziesz według niego działał, to wówczas możesz być pewny, ze żadna twoja zbiórka nie będzie nudna, a zastęp będzie ciągle się powiększał. Plan pracy składa się z:
\begin{itemize}[noitemsep] 
\item \textbf{Strony tytułowej}
\item \textbf{Ogólnej charakterystyki zastępu} --- opis sytuacji, cele długofalowe, cele na rok pracy materialne;
\item \textbf{Szczegółowej charakterystyki zastępu} --- tu zamieszczamy opis każdego przedstawiciela zastępu: wiek, szkołę, ilość lat w harcerstwie, przebieg jego służby, pasje i to, nad czym powinien popracować; charakterystyka powinna być dość szczegółowa, gdyż przecież znasz swój cały zastęp;
\item \textbf{Cele szczegółowe na rok pracy}
\item \textbf{Szczegółowy plan pracy} --- nie zapomnij, że zbiórki zastępu masz co tydzień, a raz w miesiącu zbiórkę drużyny; tu opisujesz planowany temat zbiórki i ogólnie zajęcia, jakie przeprowadzisz na niej.
\end{itemize}

\noindent
Co najważniejsze, plan powinieneś oddać do 03 IX każdego roku, aby drużynowy na podstawie planów pracy wszystkich zastępów, stworzył plan pracy drużyny. 

\section{Obrzędowść zastępu}
Każdy naród posiada swój język i własną kulturę. Właśnie to wyróżnia go spośród innych. Podobnie jest i w  harcerstwie. Mimo, że wszystkim nam  chodzi o to samo, coś sprawia że harcerze z 15 PDH, różnią się od tych z 1 ŁDH. To  coś  to  nie  tylko  barwy i numer, ale wszystko to, co jest językiem i kulturą drużyny - obrzędy.
	Po co komu obrzędy? Jest  kilka  odpowiedzi:
\begin{itemize}[noitemsep] 
\item  tradycja --- ludzie się zmieniają,  świat się zmieni,  ale obyczaje zostaną;
\item  język --- którym mówi się chłopakom o co tu  właściwie chodzi;
\item  dyscyplina --- ułatwia techniczne prowadzenie np. zbiórki czy ogniska;
\item  jedność --- my mamy swoje obrzędy;
\item  przyciąganie ---  tajemnica intryguje;
\item  i wiele innych rzeczy.
\end{itemize}
	Czy tylko drużyna może posiadać obrzędowość? Nie, oczywiście także zastępy muszą mieć swoje obyczaje. Ale:
-  muszą one współgrać z obyczajami drużyny
- nie może być  ich zbyt  dużo  aby  nie  przytłaczały
- muszą być nijako przy  okazji - chłopacy nie mogą całą zbiórkę wkuwać obrzędów
- nie wolno ich zmieniać bez powodu.

	Obrzędy i obyczaje  mogą  być  bardzo  różne:  od  sposobu  noszenia  proporca, przez  jakiś  znak  zastępu,  do  koloru  sznurowadeł  funkcyjnych zastępu  włącznie.  Ale uwaga! Obrzędowość nie  może być zbędnym balastem - musi pomagać, a nie  przeszkadzać.
	Kiedy chcesz stworzyć nowy obrzęd,  zastanów się po co chcesz to  zrobić. 